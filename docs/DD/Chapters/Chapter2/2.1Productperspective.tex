\section{Product perspective}

\textit{Expire App} is a system that provides all the functionalities described in the product functions section. It includes all the subsystems needed to fulfil these software requirements. User can access the service through the application, which should be downloadable and runnable both on iOS and Android devices either phone or tablet. All interfaces must conform to a uniform, intuitive and user-friendly design, that doesn't require the reading of detailed documentation to be used. The mobile application can run on devices that comply with the minimum requirements in terms of memory and screen size, computational power and other relevant parameters specified in the subsection 2.5. Furthermore, the back-end services are all hosted by Google’s Firebase.

\newpage

\section{External services}
In the next subsections, the main external services used in the application will be illustrated and briefly explained.

\subsection{Firebase}
\begin{wrapfigure}{R}{0.3\textwidth}
\vspace{-0.8cm}
\includegraphics[width=0.18\textwidth]{Images/external_serv/firebase.png}
\end{wrapfigure}
Firebase is a Backend-as-a-Service (Baas). It provides a variety of tools and services.It is built on Google’s infrastructure. Firebase is categorized as a NoSQLdatabase program, which stores data in JSON-like documents.
Firebase offers multiple backend services, of which Expire App uses:
\begin{itemize}
    \item Firebase Authentication: is an extensible token-based auth system and provides out-of-the-box integrations with the most common providers such as Google, Facebook, and Twitter, among others.
    \item Firestore Database: cloud-based NoSQL database server that stores and sync data, that supports automatic scaling. It stores data as documents that are logically classified into collections. The Firestore document offers support for multiple file types, numbers, strings, and nested objects.
    \item Cloud Storage: is an object storage service. In Expire App it's used for storing pictures of products
\end{itemize}


\subsection{OpenFoodFacts}
\begin{wrapfigure}{R}{0.3\textwidth}
\vspace{-0.8cm}
\includegraphics[width=0.25\textwidth]{Images/external_serv/openfoodfacts1.png}
\end{wrapfigure}

Open Food Facts is a free, online and crowdsourced database of food products from around the world licensed under the Open Database License (ODBL).
In Expire App it's involved when a user add a product by scanning barcode.
The database contains mostly food products and their details, such as nutritional values.
Those data is also used in the statistics screen, to create charts regarding nutritional values.
\newpage

\subsection{Spoonacular API}
\begin{wrapfigure}{R}{0.3\textwidth}
\vspace{-0.8cm}
\includegraphics[width=0.20\textwidth]{Images/external_serv/spoonacular.png}
\end{wrapfigure}
Spoonacular is a recipe search engine and social cooking platform. In Expire App  this service is used whenever the user tap on the recipe section.
This API provides a list of recipes based on the ingredients that the user has already saved, it's also possible to get detailed information about a recipe by tapping on it.\newline

\subsection{Google Cloud Vision API}
\begin{wrapfigure}{R}{0.3\textwidth}
\vspace{-0.8cm}
\includegraphics[width=0.20\textwidth]{Images/external_serv/vision_api.png}
\end{wrapfigure}
This service is a OCR offered by Google.
It permits the conversion of handwritten/printed texts into machine-encoded text, in Expire App it used when a user adds a new product manually, and instead of writing the list of ingredients manually, he/she can take a picture of the label, and the service will translate it into text.

\subsection{Authentication}
Firebase Authentication provides backend services. It supports authentication using passwords, phone numbers, popular federated identity providers like Google, Facebook and Twitter, and more.

\subsubsection{Google authentication}
\begin{wrapfigure}{R}{0.3\textwidth}
\vspace{-0.8cm}
\includegraphics[width=0.20\textwidth]{Images/external_serv/g.png}
\end{wrapfigure}
Authentication into Expire App using a Google account, 3rd party library is required to trigger the authentication flow, called \textit{google\textunderscore sign\textunderscore in}.
\newpage

\subsubsection{Facebook authentication}
\begin{wrapfigure}{R}{0.3\textwidth}
\vspace*{-0.8cm}
\includegraphics[width=0.13\textwidth]{Images/external_serv/f.png}
\end{wrapfigure}
Like Google authentication, also Facebook authentication required a 3rd party library, called \textit{flutter\textunderscore facebook \textunderscore auth} and some manual adjustment with native settings for android and iOS. To use Facebook authentication it's needed a Facebook account and access the Facebook Developer Platform. Facebook provides an authentication token that can be used via OAuth for authorization and log in into Firebase.\newline

\subsubsection{Apple authentication}
\begin{wrapfigure}{R}{0.3\textwidth}
\vspace{-0.8cm}
\includegraphics[width=0.13\textwidth]{Images/external_serv/a.png}
\end{wrapfigure}
Authentication into Expire App using a Apple account (possible only for iOS and iPadOS users).\newline

\subsection{Scandit}
\begin{wrapfigure}{R}{0.3\textwidth}
\vspace{-0.8cm}
\includegraphics[width=0.20\textwidth]{Images/external_serv/scandit.png}
\end{wrapfigure}
Scandit is a platform for mobile computer vision and augmented reality, it offers a variety of products, in Expire App we used the barcode scanner for reading:
\begin{itemize}
    \item barcode: when a user want to insert a new product to the list
    \item qrcode: when a user want to join a family group
\end{itemize}
\newpage