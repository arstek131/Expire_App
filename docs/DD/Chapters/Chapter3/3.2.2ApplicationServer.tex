\subsection{Application server}
The implementation of the mobile application must be autonomous from the structure of the application server. The UI must respect the design guidelines provided by Google's Material Design.
As stated before, application is developed using Flutter, which is a framework base on the Dart programming language.

To separate application states and UI, an external package called \textbf{Provider} is used.
It does mainly two jobs:
\begin{itemize}
    \item Separates state from UI
    \item Manages rebuilding UI based on state changes
\end{itemize}
Which makes loads of things simpler, from reasoning about state, to testing to refactoring. It makes codebase scalable.
It can be considered a low boiler-plate way to separate business logic from  widgets in apps.
In addition to using Providers, other external packages have been used(non-exhaustive list):
\begin{itemize}
    \item \textbf{sqflite: }SQLite plugin for Flutter
    
    \item \textbf{http: }contains a set of high-level functions and classes that make it easy to consume HTTP resources
    
    \item \textbf{flutter scandit: }barcode and qrcode scanner
    
    \item \textbf{shared preferences: }Wraps platform-specific persistent storage for simple data 
    
    \item \textbf{syncfusion flutter charts: }data visualization library charts
    
    \item \textbf{firebase: }
    \begin{itemize}
        \item \textbf{cloud firestore: }cloud-hosted, NoSQL database accessible directly via native SDKs
        \item \textbf{firebase storage: }is a service used to store and download files generated directly by clients
        \item \textbf{firebase auth: }firebase authentication API
    \end{itemize}
    
    \item \textbf{google sign in: }API to manage Google login and registration
    \item \textbf{flutter facebook auth: }API to manage Facebook login and registration
    \item \textbf{openfoodfacts: }package for the Open Food Facts API
    
    \item \textbf{testing: }
    \begin{itemize}
        \item \textbf{mockito: }testing framework
        \item \textbf{firebase auth mocks: }unit tests involving Firebase Authentication
        \item \textbf{build runner: }provides general-purpose commands for generating files, and for optionally testing the generated files or serving both source and generated files
        \item \textbf{fake cloud firestore: }Fakes to write unit tests for Cloud Firestore
        \item \textbf{google sign in mocks: }mocks for google sign in package, used for testing purposes
    \end{itemize}
    
\end{itemize}

Furthermore to get data about \texit{products} and \textit{recipes}, also external services are used.
\textbf{OpenFoodFacts }is used for retrieving information about a product given a barcode.
\textbf{Spoonacular API} is used for retrieving recipes and detail about a recipe.
